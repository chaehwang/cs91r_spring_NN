%%%%%%%%%%%%%%%%%%%%%%%%%%%%%%%%%%%%%%%%%%%%%%%%%%%%%%%%%%%%%%%
%
% Welcome to Overleaf --- just edit your LaTeX on the left,
% and we'll compile it for you on the right. If you give
% someone the link to this page, they can edit at the same
% time. See the help menu above for more info. Enjoy!
%
% Note: you can export the pdf to see the result at full
% resolution.
%
%%%%%%%%%%%%%%%%%%%%%%%%%%%%%%%%%%%%%%%%%%%%%%%%%%%%%%%%%%%%%%%
% Block diagram wire junctions
\documentclass{article}
\usepackage{tikz}
\usetikzlibrary{arrows}
\usepackage{verbatim}
\usetikzlibrary{patterns}

\begin{comment}
:Title: Block diagram line junctions
:Slug: line-junctions
:Tags: Block diagrams, Foreach, Transformations, Paths

An example of how to draw line junctions in a block diagram.
A semicircle is used to indicate that two lines are not connected.
This is a good example of how flexible TikZ' paths are.
The intersection between the lines are calculated using the convenient
``-|`` syntax. Since we want the semicircle to have its center where
the lines intersect, we have to shift the intersection coordinate
accordingly.

\end{comment}


\begin{document}

\tikzstyle{block} = [draw,minimum size=2em]
\tikzstyle{data} = [draw,shape=circle,minimum size=2em]


% diameter of semicircle used to indicate that two lines are not connected
\def\radius{.7mm}
\tikzstyle{branch}=[fill,shape=circle,minimum size=3pt,inner sep=0pt]

\begin{tikzpicture}[>=latex']
\node[data] at (0,0) (input) {I};
\node[block] at (1.5,0) (encoder) {$Encoder$};
\draw[->] (input) -- (encoder);
\node[data] at (3,0) (code) {C};
\draw[->] (encoder) -- (code);
\node[block] at (4.5,0) (decoder) {$Decoder$};
\draw[->] (code) -- (decoder);
\node[data] at (6,0) (output) {$\hat{\mathrm{I}}$};
\draw[->] (decoder) -- (output);
\end{tikzpicture}
\end{document}
