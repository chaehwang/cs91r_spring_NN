%%%%%%%%%%%%%%%%%%%%%%%%%%%%%%%%%%%%%%%%%%%%%%%%%%%%%%%%%%%%%%%
%
% Welcome to Overleaf --- just edit your LaTeX on the left,
% and we'll compile it for you on the right. If you give
% someone the link to this page, they can edit at the same
% time. See the help menu above for more info. Enjoy!
%
% Note: you can export the pdf to see the result at full
% resolution.
%
%%%%%%%%%%%%%%%%%%%%%%%%%%%%%%%%%%%%%%%%%%%%%%%%%%%%%%%%%%%%%%%
% Block diagram wire junctions
\documentclass{article}
\usepackage{tikz}
\usetikzlibrary{arrows}
\usepackage{verbatim}

\begin{comment}
:Title: Block diagram line junctions
:Slug: line-junctions
:Tags: Block diagrams, Foreach, Transformations, Paths

An example of how to draw line junctions in a block diagram.
A semicircle is used to indicate that two lines are not connected.
This is a good example of how flexible TikZ' paths are.
The intersection between the lines are calculated using the convenient
``-|`` syntax. Since we want the semicircle to have its center where
the lines intersect, we have to shift the intersection coordinate
accordingly.

\end{comment}


\begin{document}

\tikzstyle{block} = [draw,minimum size=2em]
\tikzstyle{data} = [draw,shape=circle,minimum size=2em]


% diameter of semicircle used to indicate that two lines are not connected
\def\radius{.7mm}
\tikzstyle{branch}=[fill,shape=circle,minimum size=3pt,inner sep=0pt]

\begin{tikzpicture}[>=latex']
\node[data] at (0,0) (input) {C};
\node[block] at (1.2,0) (l1) {$Linear$};
\draw[->] (input) -- (l1);
\node[block] at (2.8,0) (unpool2) {$Unpool_2$};
\draw[->] (l1) -- (unpool2);
\node[block] at (4.4,0) (dconv3) {$DConv_3$};
\draw[->] (unpool2) -- (dconv3);
\node[block] at (5.9,0) (relu3) {$ReLU$};
\draw[->] (dconv3) -- (relu3);
\node[block] at (7.4,0) (unpool1) {$Unpool_1$};
\draw[->] (relu3) -- (unpool1);
\node[block] at (9,0) (dconv2) {$DConv_2$};
\draw[->] (unpool1) -- (dconv2);
\node[block] at (10.5,0) (relu2) {$ReLU$};
\draw[->] (dconv2) -- (relu2);
\node[block] at (12,0) (dconv1) {$DConv_1$};
\draw[->] (relu2) -- (dconv1);
\node[block] at (13.5,0) (relu1) {$ReLU$};
\draw[->] (dconv1) -- (relu1);
\node[data] at (14.7,0) (output) {$\hat{\mathrm{I}}$};
\draw[->] (relu1) -- (output);

\end{tikzpicture}
\end{document}
